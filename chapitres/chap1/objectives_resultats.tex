\section{Les objectves fixées}
Le premier objectif de ce projet concerne la rédaction du cahier des charges de toutes les 
évolutions à effectuer au cours de ce projet. Ce cahier des charges contient tous les éléments 
à reprendre pour la maintenance préventive et curative, ainsi que toutes les évolutions à apporter 
à MCA pour répondre aux exigences techniques de la partie précédente.

L’objectif, en termes de délai, est de finaliser les développements fin septembre. 
Le développement concerne les quatre grandes parties du cahier des charges, soit, 
l’inventaire des matériels, les planifications, la gestion de la maintenance périodique et 
l’aperçu de l’agenda des techniciens.
\section{Les résultats obtenus}
Comme prévu, le cahier des charges et les issues sont entièrement rédigés à la fin du mois de mai 2022. 
Les développements ont commencé au début du mois de juillet de la même année.

J’ai démarré ensuite le développement du nouveau module. Sur une période d’une semaine et demi, 
j’ai initié l’architecture de la base de données ainsi que la gestion des droits, 
la récupération des données présentes en base de données et introduis la charte graphique de 
la partie Frontend du module.

Arrivé à ce point du développement, j’ai rencontré diverses difficultés qui nécessitaient des 
éclaircissements ainsi que des réponses à mes questions de la part de mes collègues. 
J’ai donc fait une pause sur cette application, et me suis mis à me documenter, faire des recherches sur bonnes pratiques à adopter.
Par la suite, j’ai commencé à affiner les développements sous la tutelle du chef produit et 
d’un collègue plus compétent sur l’application MCA.

Actuellement, nne version beta de ce module est en cours de déploiement auprès des clients de Clarisys 
Informatique pour passer la recette de test et collecter les dysfonctionnements potentiels pour améliorer les fonctionnalités de ce module.