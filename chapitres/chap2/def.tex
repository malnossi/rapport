
\section{Définition de la maintenance}
Rappelons la norme NF X 60-010 qui définit la maintenance comme étant 
\textit{« l’ensemble des activités destinées à maintenir ou rétablir un bien dans un état ou dans 
des conditions données de sureté de fonctionnement, pour accomplir une fonction requise. 
Ces activités sont une combinaison d’activités techniques, administratives et de management. »}\cite{def_maintenance}

Par bien, il faut entendre un équipement de production, un bâtiment, un logiciel, un véhicule 
ou bien le regroupement ordonné de tout cela en vue d’assurer une fonction donnée, 
le plus souvent productrice de valeur. La position du management de ce ou de ces biens 
doit être de les tenir dans un état tel qu’ils puissent, de façon optimale, effectuer les 
taches et remplir les fonctions pour lesquelles ils ont été conçus.

\section{Differents type de maintenances }
\subsection{La maintenance curative}
Dont le but est la remise en état après défaillance ou dégradation. 
Selon que la réparation est provisoire ou définitive, on parlera de palliatif (dépannage) 
ou de curatif(correctif). Il se pourra même que l’on procède à des modifications 
(améliorations) en vue d’éviter que la panne ne se reproduise ou en vue de faciliter 
les interventions.
\pagebreak
\subsection{La maintenance préventive}
Qui s’attache par l’ensemble des travaux d’actions de surveillance à réduire la probabilité 
d’apparition de la défaillance ou de la dégradation du service.

Si les activités de préventif sont réalisées selon un calendrier fixe, il s’agit d’un préventif 
systématique. Si, en revanche, ces activités sont fonction d’observations ponctuelles faites sur l’état du bien, 
il s’agira de maintenance conditionnelle.

Enfin, si c’est l’évolution d’un phénomène que l’on suit afin de prévoir quand il faudra intervenir ou prédire dans le temps 
quand la dégradation irréversible se produira ou deviendra intolérable, on parlera de prévisionnel ou prédictif.

Toutes ces opérations servent à produire de la disponibilité, à donner au bien des points de rendement, à le conserver
 en état de bon fonctionnement. Parce que ces opérations devaient être assurées au meilleur coût, l’entreprise fut amenée 
 progressivement à gérer la fonction maintenance de façon plus précise et plus rigoureuse. Bien gérer, c’est connaitre et 
 décider en toute connaissance.

