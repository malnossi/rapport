\section{G.M.A.O}
GMAO signifie la \textit{Gestion de la Maintenance Assistée par Ordinateur}. C’est un progiciel qui contient toutes les données des opérations de maintenance curatives ainsi que préventives réalisées et en prévision. 
Mais surtout, il facilite la communication et les prises de décisions, grâce à une mémoire informatisée et des flux paramétrés.
Le logiciel GMAO vient soutenir les équipes de maintenance dans leur travail. 
Il facilite le suivi des activités, fiabilise les machines, gère les risques, permet de connaître le nombre d’arrêts et leurs causes, de planifier les maintenances. En d’autres termes, il permet de mieux gérer la production.

\vspace*{1cm}
Les bénéfices d’une GMAO :\\
o	Assurer une visibilité optimale des équipements\\
o	Assurer la suivie des opérations et des interventions sur les équipements.\\
o	Assurer l’enregistrement et la traçabilité des opérations.\\
o	Reporting et analyse des statistiques.\\
o	 Meilleur communication interne\\

Le rôle d’une GMAO au sein de laboratoire :\\
o	Connaître et identifier et gérer les équipements à maintenir : inventaire, localisation,\\
o	Gestion de l’information relative à l’équipement : document constructeur, contrat de prestation etc.\\
o	Piloter la maintenance : préventive, curative, corrective, améliorative,\\
o	Planifier les interventions,\\
o	Gérer les demandes d’intervention,\\
o	Coordoner le personnel et les plannings : activités, métiers, plan de charge, prévisions, etc.\\
\pagebreak