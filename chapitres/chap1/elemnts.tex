\section{Élements pour répondre aux éxigences de la norme NF en ISO-15189}
La gestion de l’équipement est l’un des points essentiels du système de gestion de la qualité.
Une bonne gestion de l’équipement au laboratoire est nécessaire pour assurer la justesse, 
la fiabilité et la pertinence des analyses.


Les bénifices d'une bonne gestion des équipements sont nombreux:\\
•	Aide à maintenir un haut niveau de fonctionnement du laboratoire\\
•	Réduit les variations entre les résultats des tests, et augmente la confiance du technicien dans la justesse des résultats\\
•	Diminue les frais de réparation : moins de réparations seront nécessaires sur un équipement bien entretenu\\
•	Augmente la durée de vie des instruments\\
•	Réduit les interruptions de fonctionnement dues à des pannes et des défauts\\
•	Augmente la sécurité pour les employés\\
•	Permet une meilleure satisfaction du client\\

Le laboratoire doit définir clairement une politique relative à la gestion de la maintenance et 
celle de la documentation afférente. La durée de conservation des enregistrements en fonction de 
leur nature est précisée dans un document. Ces éléments peuvent être classés en trois groupes :\\
•	Identification des matériels incluant les analyseurs ;\\
•	Documentation produite par le fournisseur, conformément à la directive européenne ;\\
•	Documentation préparée par le laboratoire incluant toutes les procédures en lien direct avec la maintenance de l'analyseur.\\
\pagebreak
