\chapter{Conclusion}
Cette deuxième année d’apprentissage au sein de la société Clarisys Informatique m’a permis de développer des compétences techniques, particulièrement sur les langages Python et JavaScript.

Grâce aux différents développements que j’ai pu effectuer dans le cadre de l’évolution du middleware MCA, j’ai pu acquérir une certaine expérience dans le développement web FullStack. 

De ce fait, la plupart des tâches qui me sont confiées aujourd’hui sont dans la continuité de mon projet. 

J’ai donc effectué de nombreuses modernisations de code et des nouvelles interfaces pour une meilleure expérience utilisateur, notamment via ce nouveau projet qui est davantage orienté vers l’UX.

De plus, j’ai pu acquérir à la fois de solides connaissances sur le monde de la biologie médicale et les enjeux généraux du système de santé français mais aussi les compétences nécessaires à un ingénieur, telles que la capacité d’analyse, la mobilisation de ressources techniques, l’utilisation d’outils de modélisation...

Tout au long de ce projet, j’ai pu avoir un aperçu du métier de développeur FullStack, des enjeux de sécurité, de l’interopérabilité ainsi que le point de vue législatif de la biologie médicale.

D’autre part, en ce qui concerne mon futur au sein de Clarisys Informatique, j’aspire à me spécialiser sur le Framework Vuejs. L’équipe technique l’a élu en tant que le Framework Frontend à adopter pour la nouvelle interface utilisateur.

 Pour ce qui est de mon projet professionnel à plus long terme, je souhaiterais idéalement rester dans le domaine de la santé et être au plus proche du métier, ce qui est le cas à Clarisys Informatique, tout en travaillant en tant que développeur FullStack. 

J’aimerais également être amené à travailler avec des outils modernes et en perpétuelle évolution tels que Django et VueJs.
