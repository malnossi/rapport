\chapter*{}
\begin{center}
    \large\textbf{Introduction}
\end{center}
\lettrine[findent=2pt]{{\textbf{L}}}{ }’importance de la fonction maintenance dans les laboratoires d’analyse médicale, prend une nouvelle dimension et pousse les dirigeants dans une recherche accrue de compétitivité, de sécurité, de rentabilité, de modernisation et de disponibilité.
Ainsi, la nécessité d’informatiser cette fonction apparait comme une évidence, que ce soit sur les plans économiques, techniques ou bien organisationnels.
De nombreuses applications ont vu le jour au fur et à mesure que les besoins se sont fait ressentir avec des spécificités particulières, il faut alors envisager : une conception modulaire ou les différentes fonctions seront séparées ; une conception ouverte c'est-à-dire une conception évolutive permettant d’intégrer d’autres fonctions futures ; une conception intégrée avec une base de données unique qui permet d’établir le lien entre les nombreuses informations de maintenance ; un interfaçage possible avec d’autres systèmes ; une utilisation pour tous et sur le terrain ; une possibilité d’accès par code aux différents niveaux du logiciel. 
La GMAO (Gestion de Maintenance Assistée par Ordinateur) devient alors un outil de référence indispensable pour la gestion et la traçabilité, dans les laboratoires soucieux de gérer la maintenance et le flux d’information engendré par celles-ci.
Mon travail consiste à réaliser une application de GMAO dont le but sera d’optimiser la gestion de la maintenance afin de remédier à tous les problèmes rencontrés par les services concernés, et d’atteindre les objectifs fixés par les responsables.
Le premier chapitre présente des généralités sur la maintenance : définition de la maintenance, différentes politiques, niveaux, formes organisationnelles, outils, organisation des opérations et coûts de cette dernière. 
Le deuxième chapitre introduit et défini la GMAO d’une façon générale au départ puis propose, par la suite, un exemple d’une GMAO appliquée dans les laboratoires ainsi que ses différents modules analysés.
Le troisième chapitre présente les démarches à suivre pour l’informatisation de la maintenance puis se consacre en particulier à la présentation de l’application.
\thispagestyle{empty}