\section{Les problèmatiques}
L’ISO 15189 est une norme internationale publiée par l’ISO en 2012 qui spécifie les exigences de qualité et de compétence propres aux laboratoires de biologie médicale (LBM1). 
Son titre est "Laboratoire de biologie médicale. Exigences concernant la qualité et la compétence" Cette norme est spécifique aux LBM à la différence de la norme ISO/CEI 17025 qui concerne tous les laboratoires d'étalonnages et d'essais.
Le système de management de la qualité des laboratoires accrédités 15189 est fondé sur la norme ISO 9001:2008. La version de décembre 2012 constitue une révision technique de la version précédente, de 2007.

En France, le Cofrac accrédite sur demande les laboratoires d'analyses de biologie médicales selon un référentiel qui leur est propre. Cette accréditation est une reconnaissance des compétences techniques. Dans le cadre de la nouvelle loi HPST, cette accréditation est devenue obligatoire.

En 2013, tout laboratoire, public ou privé, devait démontrer qu'il avait entrepris une démarche d'accréditation et en 2016, tous les laboratoires publics ou privés de France devaient être accrédités ISO 15189 sur au moins 50 % des examens (70 % en 2018 et 100% en 2020)
Par ailleurs, depuis 2011, un laboratoire de biologie médicale qui souhaiterait ouvrir ses portes ne pourra pas invoquer de délai et devra être accrédité à l'ouverture.
Concernant les équipements de laboratoires médicaux, et dans le chapitre 3.5 intitulé « Matériel de laboratoire » de la norme. Extrait du guide de bonne exécution des analyses de biologie médicale : \textit{« Les appareils doivent être périodiquement et efficacement inspectés, nettoyés, entretenus et vérifiés selon la procédure en vigueur. L'ensemble de ces opérations ainsi que les visites d'entretien et de réparation du constructeur ou de l'organisme de maintenance doivent être consignées par écrit dans un registre de maintenance affecté à chaque instrument. Le biologiste-responsable du laboratoire doit s'assurer de la mise en œuvre des moyens métrologiques nécessaires à leur vérification usuelle. Les notices d'utilisation et de maintenance d'appareils doivent être mises en permanence à la disposition du personnel utilisateur et respectées. Le fonctionnement des appareils doit être vérifié selon la fréquence préconisée par le fabricant.
Des procédures de remplacement doivent être prévues en cas de dysfonctionnement d'un automate : mise en œuvre d'autres techniques ou transmission des échantillons biologiques à un autre laboratoire de biologie médicale. Ces procédures doivent prendre en compte la transférabilité des résultats afin de conserver un historique cohérent des données du patient. Responsabilité de la personne chargée de l'assurance de qualité.
L'organisation du système d'assurance de qualité du laboratoire de biologie médicale peut être déléguée par le biologiste-responsable du laboratoire à un biologiste médical ou à une personne chargée de la gestion du système d'assurance de qualité qui devra avoir la formation, la compétence et l'expérience nécessaires pour accomplir la tâche qui lui sera confiée. Elle doit notamment s'assurer : [...] – de la maintenance, du bon fonctionnement des appareillages ; [...]. »}\cite{guide_maint}
\pagebreak