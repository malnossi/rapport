\begin{abstract}
    \lettrine[findent=2pt]{{\textbf{L}}}{} 'ère de l’entretien est révolue depuis longtemps et la maintenance 
structurée dans le secteur tertiaire est aujourd’hui un bonus. Celle-ci,
ne peut se faire aisément sans une GMAO paramétrable en standard, 
adaptable aux flux fonctionnels de chaque organisation. 
La maintenance des analyseurs et la documentation associée constituent une exigence de la norme NF EN ISO 15189 et de la réglementation en vigueur et, à ce titre, elles doivent être définies dans la politique qualité du laboratoire. La gestion de la maintenance préventive périodique doit être mise en œuvre par les responsables du laboratoire et la maintenance corrective doit être organisée pour ne pas compromettre la continuité des tâches au sein de la structure. Les différentes recommandations concernent l'identification des matériels incluant les analyseurs, les conditions environnementales indispensables pour un bon fonctionnement de l'analyseur, la documentation produite par le fournisseur et celle préparée par le laboratoire incluant toutes les procédures en lien direct avec la maintenance de l'analyseur.
\end{abstract}